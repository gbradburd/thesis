
% Their are two abstracts. One that is published externally from your
% dissertation, and one that is internal. Of course, the text of the
% abstract will be the same. So, we define a macro to hold the body of our
% abstract.
% at 345 words - With electronic filing there is no longer a word limit

\newcommand{\myabstract}{
\par Genetic variation provides the raw materials for both local adaptation and the formation of new species, 
and understanding the processes generating and maintaining the diversity of living organisms 
is a fundamental aim across all areas and levels of biological research. 
The central question that motivated this dissertation research is: 
what are the patterns of variation within and between populations and species, 
and what processes, ecological and evolutionary, are generating those patterns?
%Spatial patterns of population structure are the product of a complex history of demography and migration, 
%and can provide valuable insights into the ecology of an organism. 
My dissertation is focused on developing statistical methods to infer and visualize spatial patterns of genetic differentiation, 
and provide clues about the processes that have generated them.
%%
\par In my first chapter, I develop a statistical method to quantify 
the relative contributions of ecological and geographic distance to patterns of genetic differentiation.
The method, BEDASSLE,
%Bayesian Estimation of Differentiation in Alleles by Spatial Structure and Local Ecology (BEDASSLE), 
models the allele frequencies in a set of populations at a set of unlinked loci 
as spatially correlated Gaussian processes, in which the covariance structure 
is a decreasing function of both geographic and ecological distance.
%%
\par In my second chapter, I discuss the conceptual framework of Isolation by Environment 
(IBE, as opposed to Isolation by Distance, or IBD) in a review and synthesis.
This work clearly defines IBE as a pattern, rather than a process, 
and details the different processes that can generate a pattern of IBE.
%%
\par In my third chapter, I extend the statistical framework developed in the first chapter
to quantify signals of IBD and IBE to a general method for 
inferring and visualizing patterns of population structure. 
This method, SpaceMix, infers, for a set of sequenced samples, 
a map in which the distances between population locations reflect genetic, 
rather than geographic, proximity.
%%
\par Together, these chapters represent an advance in the conceptual 
and statistical framework foranalyzing spatial patterns 
%and statistical framework for inferring, analyzing, and visualizing spatial patterns 
of population genetic structure across landscapes.}
%The partial Mantel test has traditionally been used to address this question; however, it was recently demonstrated to have a pathological type 1 error rate when the environment is spatially autocorrelated. The method I developed, Bayesian Estimation of Differentiation in Alleles by Spatial Structure and Local Ecology (BEDASSLE), models the allele frequencies in a set of populations at a set of unlinked loci as spatially correlated Gaussian processes, in which the covariance structure is a decreasing function of both geographic and ecological distance. I demonstrated the correct statistical behavior of this method using simulations, and applied it in teosinte, in which I estimated the contribution of elevation to population structure, and in humans, in which I estimated the effect size of the Himalayas on genetic differentiation among Eurasians. I coded this method and released it as an R package for open use by other empirical researchers.

%In my second chapter, I extended the conceptual framework of Isolation by Environment (IBE, as opposed to Isolation by Distance, or IBD) in a review and synthesis. This work clearly defined IBE as a pattern, rather than a process, and detailed the different processes that can generate a pattern of IBE. I also provided a �field-guide� of the statistical methods available for quantifying IBE and considerations in designing an appropriate sampling scheme to identify its signal (and disentangle if from that of IBD).

%In my third chapter, I extended the statistical framework I used previously to quantify signals of IBD and IBE to a general method for inferring and visualizing patterns of population structure. This method, SpaceMix, infers, for a set of sequenced samples, a map in which the distances between population locations reflect genetic, rather than geographic, proximity. The result is a �geogenetic� map in which the distances between populations are effective distances, indicative of the way that populations perceive the distances between themselves. Nearby populations that are genetically dissimilar (e.g. separated by a barrier) may have distant geogenetic locations, while two distant populations that are closely related (e.g., the parent and daughter populations of a recent expansion) may be geogenetic neighbors. In this spatial context, �admixture� can be thought of as the outcome of unusually long-distance gene flow; it results in relatedness between populations that is anomalously high given the distance that separates them. SpaceMix depicts the effect of admixture using arrows, from a source of admixture to its target, on the inferred map. The inferred geogenetic map is an intuitive and information-rich visual summary of patterns of population structure.
